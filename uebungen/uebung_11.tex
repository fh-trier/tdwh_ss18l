\section{Übung - XML}
\label{sec:uebung_11}

% ##########################################################################
% ############################### Aufgabe 01 ###############################
% ##########################################################################
\subsection{Aufgabe}
\label{subsec:uebung_11.aufgabe_01}

\subsubsection*{Lösung}
\label{subsubsec:uebung_11.aufgabe_01.loesung}
\inputsql{./loesungen/uebung_11/uebung_11_-_aufgabe_01.sql}


% ##########################################################################
% ############################### Aufgabe 02 ###############################
% ##########################################################################
\subsection{Aufgabe}
\label{subsec:uebung_11.aufgabe_02}
Füllen Sie diese mit den Werten aus der Tabelle DWH.Departments. Sie können dabei folgendes SQLX-Statement nutzen um die XML-Strukturen zu erzeugen.

\subsubsection*{Lösung}
\label{subsubsec:uebung_11.aufgabe_02.loesung}
\inputsql{./loesungen/uebung_11/uebung_11_-_aufgabe_02.sql}


% ##########################################################################
% ############################### Aufgabe 03 ###############################
% ##########################################################################
\subsection{Aufgabe}
\label{subsec:uebung_11.aufgabe_03}
Prüfen Sie auf den XML-Dokumenten ob es Departments gibt, die keinen Manager besitzen? Falls ja, welche sind es?

\subsubsection*{Lösung}
\label{subsubsec:uebung_11.aufgabe_03.loesung}
\inputsql{./loesungen/uebung_11/uebung_11_-_aufgabe_03.sql}


% ##########################################################################
% ############################### Aufgabe 04 ###############################
% ##########################################################################
\subsection{Aufgabe}
\label{subsec:uebung_11.aufgabe_04}
Geben Sie die Nummer und den Namen  aller Departments aus, die in Toronto liegen. Nutzen Sie dazu XMLQuery und das FLWOR Konstrukt.

\subsubsection*{Lösung}
\label{subsubsec:uebung_11.aufgabe_04.loesung}
\inputsql{./loesungen/uebung_11/uebung_11_-_aufgabe_04.sql}


% ##########################################################################
% ############################### Aufgabe 05 ###############################
% ##########################################################################
\subsection{Aufgabe}
\label{subsec:uebung_11.aufgabe_05}
Erzeugen Sie eine XMLTable, die alle Elemente und Attribute der XML-Dokumente bereitstellt.

\begin{table}[H]
  \centering
  \ttfamily
  \begin{tabular}{|l|l|l|l|}
    \hline
    \textbf{DEPTNO} & \textbf{NAME}    & \textbf{MANAGER} & \textbf{CITY} \\
    \hline
    70              & Public Relations & Baer             & Munich        \\
    50              & Shipping         & Fripp South      & San Francisco \\
    100             & Finance          & Greenberg        & Seattle       \\
    20              & Marketing        & Hartstein        & Toronto       \\
    \hline
  \end{tabular}
\end{table}

\subsubsection*{Lösung}
\label{subsubsec:uebung_11.aufgabe_05.loesung}
\inputsql{./loesungen/uebung_11/uebung_11_-_aufgabe_05.sql}