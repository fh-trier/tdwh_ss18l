\section{Übung - Analytische Funktionen}
\label{sec:uebung_07}

% ##########################################################################
% ############################### Aufgabe 01 ###############################
% ##########################################################################
\subsection{Aufgabe}
\label{sec:uebung_07.aufgabe_01}
Zeige für jede Bestellposition die insgesamt in der Produktoberkategorie (Parentcategory) abgesetzte Menge.

\begin{figure}[H]
  \centering
  \includegraphics[width=0.8\textwidth]{img//uebung_07_-_aufgabe_01.png}
  \label{img:uebung_07_-_aufgabe_01}
\end{figure}

\subsubsection*{Lösung}
\label{sec:uebung_07.aufgabe_01.loesung}
\inputsql{./loesungen/uebung_07/uebung_07_-_aufgabe_01.sql}


% ##########################################################################
% ############################### Aufgabe 02 ###############################
% ##########################################################################
\subsection{Aufgabe}
\label{sec:uebung_07.aufgabe_02}
Klassifiziere die Departments nach dem durchschnittlichen Gehalt der Mitarbeiter, die in jenem Department arbeiten, in drei Klassen.

\begin{table}[H]
  \centering
  \ttfamily
  \begin{tabular}{|l|l|l|l|}
    \hline
    \textbf{DEPT\_ID} & \textbf{DEPT\_NAME}  & \textbf{AVG\_SAL} & \textbf{CLASS} \\
    \hline
    140               & Control And Credit   & 0                 & 1              \\
    260               & Recruiting           & 0                 & 1              \\
    150               & Shareholder Services & 0                 & 1              \\
    160               & Benefits             & 0                 & 1              \\
    $[$\dots$]$       &                      &                   &                \\
    20                & Marketing            & 9500              & 3              \\
    70                & Public Relations     & 10000             & 3              \\
    110               & Accounting           & 10154             & 3              \\
    90                & Executive            & 19333,33          & 3              \\
    \hline
  \end{tabular}
\end{table}

\subsubsection*{Lösung}
\label{sec:uebung_07.aufgabe_02.loesung}
\inputsql{./loesungen/uebung_07/uebung_07_-_aufgabe_02.sql}


% ##########################################################################
% ############################### Aufgabe 03 ###############################
% ##########################################################################
\subsection{Aufgabe}
\label{sec:uebung_07.aufgabe_03}
Welche sind die Top 4 Produktkategorien, gemessen am Umsatz, für alle Bestellungen aus dem Jahr 2014 die an Kunden aus Brasilien gingen?

\begin{table}[H]
  \centering
  \ttfamily
  \begin{tabular}{|l|l|l|l|}
    \hline
    \textbf{P\_CATEGORY\_NAME} & \textbf{CATEGORY\_NAME}  & \textbf{UMSATZ} & \textbf{RANG} \\
    \hline
    hardware & hardware8 & 7228652 & 1 \\
    software & software6 & 7145551 & 2 \\
    hardware & hardware4 & 5216302 & 3 \\
    hardware & hardware3 & 5051191 & 4 \\
    \hline
  \end{tabular}
\end{table}

\subsubsection*{Lösung}
\label{sec:uebung_07.aufgabe_03.loesung}
\inputsql{./loesungen/uebung_07/uebung_07_-_aufgabe_03.sql}


% ##########################################################################
% ############################### Aufgabe 04 ###############################
% ##########################################################################
\subsection{Aufgabe}
\label{sec:uebung_07.aufgabe_04}
Liste die Umsätze der einzelnen Monate für 2015 auf. Dabei sollen zusätzlich die laufende Summe und der durchschnittliche Umsatz, berechnet aus dem vorherigen, aktuellen sowie nachfolgendem Monat, ausgegeben werden.

\begin{table}[H]
  \centering
  \ttfamily
  \begin{tabular}{|l|r|r|r|}
    \hline
    \textbf{MONAT}  & \textbf{UMSATZ €} & \textbf{LAUF\_SUMME €}  & \textbf{AVG\_UMSATZ €}  \\
    \hline
    2015-01         & 13008688          & 13008688                & 12598426,50             \\
    2015-02         & 12188165          & 25196853                & 12509253,67             \\
    2015-03         & 12330908          & 37527761                & 11674520,67             \\
    2015-04         & 10504489          & 48032250                & 12901664,33             \\
    2015-05         & 15869596          & 63901846                & 12807859,00             \\
    2015-06         & 12049492          & 75951338                & 13661033,33             \\
    2015-07         & 13064012          & 89015350                & 13962507,33             \\
    2015-08         & 16774018          & 105789367               & 14231523,00             \\
    2015-09         & 12856539          & 118645907               & 16080042,67             \\
    2015-10         & 18609571          & 137255478               & 15068229,33             \\
    2015-11         & 13738578          & 150994056               & 16088294,33             \\
    2015-12         & 15916734          & 166910790               & 14827656,00             \\
    \hline
  \end{tabular}
\end{table}


\subsubsection*{Lösung}
\label{sec:uebung_07.aufgabe_04.loesung}
\inputsql{./loesungen/uebung_07/uebung_07_-_aufgabe_04.sql}


% ##########################################################################
% ############################### Aufgabe 05 ###############################
% ##########################################################################
\subsection{Aufgabe}
\label{sec:uebung_07.aufgabe_05}
Erzeuge eine Ausgabe mit den Spalten \texttt{COUNTRY\_NAME}, \texttt{CHANNEL\_DESC}, \texttt{SALES\$}, \texttt{RANG} indem für die Vertriebskanäle alle Umsätze, die darüber gemacht wurden von März bis Juni 2008 aufgelistet und in der Spalte \texttt{RANG} den Rang innerhalb der Vertriebskanäle für das jeweilige Land vermerkt ist.

\begin{table}[H]
  \centering
  \ttfamily
  \begin{tabular}{|l|l|l|l|}
    \hline
    \textbf{COUNTRY} & \textbf{CHANNEL\_DESC} & \textbf{SALES\$} & \textbf{RANG}  \\
    \hline
    Argentina        & Direct Sales           & 291299           & 1              \\
    Argentina        & Partners               & 100131           & 2              \\
    Argentina        & Catalog                & 57064            & 3              \\
    Australia        & Tele Sales             & 717783           & 1              \\
    Australia        & Internet               & 508367           & 2              \\
    Australia        & Partners               & 452358           & 3              \\
    Australia        & Catalog                & 175854           & 4              \\
    Australia        & Direct Sales           & 47058            & 5              \\
    Belgium          & Partners               & 661416           & 1              \\
    Belgium          & Catalog                & 448693           & 2              \\
    Belgium          & Direct Sales           & 144147           & 3              \\
    $[$\dots$]$      &                        &                  &                \\
    \hline
  \end{tabular}
\end{table}

\subsubsection*{Lösung}
\label{sec:uebung_07.aufgabe_05.loesung}
\inputsql{./loesungen/uebung_07/uebung_07_-_aufgabe_05.sql}


% ##########################################################################
% ############################### Aufgabe 06 ###############################
% ##########################################################################
\subsection{Aufgabe}
\label{sec:uebung_07.aufgabe_06}
Welche Produkte wurden häufiger abgesetzt als der Durchschnitt der Produkte innerhalb der jeweiligen Produktoberkategorie (Hardware, Software,…)?

\subsubsection*{Lösung}
\label{sec:uebung_07.aufgabe_06.loesung}
\inputsql{./loesungen/uebung_07/uebung_07_-_aufgabe_06.sql}
