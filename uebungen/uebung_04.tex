\section{Übung - SCD}
\label{sec:uebung_04}
Gegeben sei folgender Aussschnitt aus einem bestehenden DWH:

\begin{table}[H]
  \centering
  \ttfamily
  \small
  \begin{tabular}{|l|l|l|l|l|}
    \hline
    \textbf{KundeDIMKey}  & \textbf{Menge} & \textbf{\dots}       \\
    \hline
    1                     & 7              &                      \\
    1                     & 3               &                     \\
    \hline
  \end{tabular}
  \caption{Faktentabelle}
\end{table}

\begin{table}[H]
  \centering
  \ttfamily
  \small
  \begin{tabular}{|l|l|l|l|l|}
    \hline
    \textbf{KundeDIMKey}          & \textbf{KundeId}                & \textbf{Nachname}                 & \textbf{Geburt}                       \\
    \hline
    \textbf{\textcolor{blue}{1}}  & \textbf{\textcolor{blue}{K01A}} & \textbf{\textcolor{blue}{Meyer}}  & \textbf{\textcolor{blue}{01.08.68}}   \\
    2                             & K02A                            & Bauer                             & 03.09.85                              \\
    \hline
  \end{tabular}
  \caption{Dimensionstabelle Kunde}
\end{table}

% ############################### Aufgabe 5a ##############################
\subsection{SCD-Typ 1}
\label{sec:uebung_05.aufgabe_1a}

\begin{table}[H]
  \centering
  \ttfamily
  \small
  \begin{tabular}{|l|l|l|l|}
    \hline
    \textbf{KundeDIMKey}  & \textbf{KundeId}  & \textbf{Nachname}                 & \textbf{Geburt} \\
    \hline
    1                     & K01A              & \textbf{\textcolor{red}{Schmitt}} & 01.08.68        \\
    2                     & K02A              & Bauer                             & 03.09.85        \\
    \hline
  \end{tabular}
  \caption{SCD-Typ 1 - Änderung überschreiben}
\end{table}

% ############################### Aufgabe 5b ##############################
\subsection{SCD-Typ 2}
\label{sec:uebung_05.aufgabe_1b}

\begin{table}[H]
  \centering
  \ttfamily
  \small
  \begin{tabular}{|l|l|l|l|l|l|l|}
    \hline
    \textbf{KundeDIMKey} & \textbf{KundeId} & \textbf{Nachname} & \textbf{Geburt} & \textbf{EffDatum} & \textbf{ExpDatum} & \textbf{CRIndikator} \\
    \hline
    1 & K01A  & \textbf{\textcolor{red}{Meyer}} & 01.08.68 & \textbf{\textcolor{red}{01.02.2007}} & \textbf{\textcolor{red}{22.05.2016}} & \textbf{\textcolor{red}{FALSE}} \\
    2 & K02A & Bauer & 03.09.85 & 01.01.05 & & TRUE \\
    \textbf{\textcolor{red}{3}} & \textbf{\textcolor{red}{K01A}}  & \textbf{\textcolor{red}{Schmitt}} & \textbf{\textcolor{red}{01.08.68}} & \textbf{\textcolor{red}{23.05.2016}} &  & \textbf{\textcolor{red}{TRUE}} \\
    \hline
  \end{tabular}
  \caption{SCD-Typ 2 - Neue Zeile einfügen}
\end{table}

% ############################### Aufgabe 5c ##############################
\subsection{SCD-Typ 3}
\label{sec:uebung_05.aufgabe_1c}

\begin{table}[H]
  \centering
  \ttfamily
  \small
  \begin{tabular}{|l|l|l|l|l|}
    \hline
    \textbf{KundeDIMKey}  & \textbf{KundeId}  & \textbf{Nachname}                 & \textbf{Geburt} & \textbf{NachnameAlt}            \\
    \hline
    1                     & K01A              & \textbf{\textcolor{red}{Schmitt}} & 01.08.68        & \textbf{\textcolor{red}{Meyer}} \\
    2                     & K02A              & Bauer                             & 03.09.85        &                                 \\
    \hline
  \end{tabular}
  \caption{SCD-Typ 3 - Neues Attribut einfügen}
\end{table}