\section{Übung - Implementierung}
\label{sec:uebung_05}

% ##########################################################################
% ############################### Aufgabe 01 ###############################
% ##########################################################################
\label{subsec:uebung_05.aufgabe_01}
\subsection{Aufgabe}
Sie sind beteiligt an der Implementierung eines Core Data Warehouse nach dem Star Schema auf Grundlage der Tabellen des DWH-Users. Alle Core DWH Tabellen sind bereits angelegt, mit Ausnahme der Customer Dimension und der Faktentabelle.

Führen Sie das Skript 04.2-Implementierung\_Skript.sql aus dem Vorlesungsverzeichnis aus, um die notwendige Tabellenstruktur in Ihrem Schema bereit zu stellen und legen Sie anschließend alle fehlenden Tabellen, Constraints usw. an. Letztlich sollen die von Ihnen erstellten Tabellen gefüllt werden. Greifen Sie dazu auf die OLTP-Tabellen aus dem Quellsystem zu.

\begin{WARN}
 Lösung fehlt!
\end{WARN}

%\subsubsection*{Lösung}
%\label{subsubsec:uebung_05.aufgabe_01.loesung}
%\inputsql{./loesungen/uebung_05/uebung_05_-_aufgabe_01.sql}

% ##########################################################################
% ############################### Aufgabe 02 ###############################
% ##########################################################################
\label{subsec:uebung_05.aufgabe_02}
\subsection{Aufgabe}
Die Vertriebsabteilung benötigt einen stets aktuellen Report, der den Absatz je Produktunterkategorie für jedes Jahr, Quartal und Mitarbeiter ausgibt. Die Datenbasis des Reports soll als eigene Tabelle implementiert werden.

Nutzen Sie hierzu Ihr komplettiertes Core Date Warehouse aus Aufgabe 1. Alternativ können Sie auch auf die Tabellen des DWH's Schemas zugreifen.

\begin{WARN}
 Lösung fehlt!
\end{WARN}

% ##########################################################################
% ############################### Aufgabe 03 ###############################
% ##########################################################################
\label{subsec:uebung_05.aufgabe_03}
\subsection{Aufgabe}
Die Antwortzeiten der folgenden Abfragen sind zu hoch. Erstellen Sie eine geeignete Materialized View. Die Abfragen sollen dabei nicht verändert werden. (Sie müssen die entsprechenden Tabellennamen / Attributsnamen natürlich an Ihr Schema anpassen.)

\subsubsection*{Lösung}
\label{subsubsec:uebung_05.aufgabe_03.loesung}
\inputsql{./loesungen/uebung_05/uebung_05_-_aufgabe_03.sql}















