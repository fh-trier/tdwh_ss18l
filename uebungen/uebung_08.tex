\section{Übung - AWK}
\label{sec:uebung_08}
Die Datei awk\_cust\_awk.csv im Verzeichnis csv hat folgenden Aufbau:

\begin{minted}{bash}
time;first;last;address;custvalue;custorders;id
11:18:40 AM;allyson;dANIELS;2835 Schorsch;1617,39;21;1
6:22:59 AM;LUNA;morse;9627 Pittsburgh;803,25;18;2
9:06:12 PM;Jake;POOLE;6599 Midway Airport;271.46;13;3
\end{minted}

% ##########################################################################
% ############################### Aufgabe 01 ###############################
% ##########################################################################
\label{subsec:uebung_08.aufgabe_01}
\subsection{Aufgabe}
Überprüfen Sie die CSV-Datei auf folgende Fehler und schreiben Sie die fehlerhaften Datensätze in eine Datei error\_cust\_data.csv. Beachten Sie dabei, dass in der ersten Zeile Spaltenüberschriften stehen und diese nicht geprüft werden müssen.

\begin{itemize}
\item Jede Zeile enthält genau 7 Felder
\item Die Felder first und last bestehen ausschließlich aus Alphazeichen
\item Die Uhrzeit im Feld time hat folgendes Format: hh:mm:ss AM/PM
  \begin{itemize}
    \item 1 oder 2-stellige Stundenangabe 1-12,
    \item 2 stellige Minuten- und Sekundenangaben 0-59
    \item gefolgt von einem Leerzeichen und AM oder PM
    \item z.B.: 9:18:56 PM oder 11:00:00 AM
  \end{itemize}
\end{itemize}


\subsubsection*{Lösung}
\label{subsubsec:uebung_08.aufgabe_01.loesung}
\inputawk{./awk/uebung_08_-_aufgabe_01.awk}


% ##########################################################################
% ############################### Aufgabe 02 ###############################
% ##########################################################################
\label{subsec:uebung_08.aufgabe_01}
\subsection{Aufgabe}
Erzeugen Sie mit AWK folgende Ausgabe, bei der die Spaltenüberschrift und die ersten 4 Kunden gefolgt von `………` ausgegeben werden.

Anschließend folgt:
\begin{itemize}
  \item die Anzahl der Kunden
  \item der kleinste Kundenwert
  \item die Anzahl der insgesamt georderten und der durchschnittlich je Kunde georderten Bestellungen
\end{itemize}

\subsubsection*{Lösung}
\label{subsubsec:uebung_08.aufgabe_02.loesung}
\inputawk{./awk/uebung_08_-_aufgabe_02.awk}


% ##########################################################################
% ############################### Aufgabe 03 ###############################
% ##########################################################################
\label{subsec:uebung_08.aufgabe_01}
\subsection{Aufgabe}
Stellen Sie folgendes sicher (z.B. durch ersetzen der Werte) und überführen Sie die Datei in eine korrigierte CSV-Datei „cust\_bereinigt.csv“, in der alle Felder durch Komma getrennt sind.

\begin{itemize}
  \item Sorgen Sie dafür, dass jeder Vorname und Nachname mit einem Großbuchstaben beginnt und anschließend nur Kleinbuchstaben folgen. (Gehen Sie hier davon aus, dass Vor- und Nachnamen ausschließlich aus Alphazeichen bestehen. Es muss keine Ersetzung eventueller Ziffern oder Sonderzeichen in den Namen durchgeführt werden)
  \item Übertragen Sie die Uhrzeit in das 24 Stunden-Format z. B.: aus `9:08:31 PM` wird `21:08:31` aus `7:45:13 AM` wird `7:45:13`
\end{itemize}


\subsubsection*{Lösung}
\label{subsubsec:uebung_08.aufgabe_03.loesung}
\inputawk{./awk/uebung_08_-_aufgabe_03.awk}


% ##########################################################################
% ############################### Aufgabe 04 ###############################
% ##########################################################################
\label{subsec:uebung_08.aufgabe_04}
\subsection{Aufgabe}
Bilde folgende Anforderungen durch einen regulären Ausdruck ab:

\begin{itemize}
  \item Eine Postleitzahl soll aus 4 bis 6 Ziffern bestehen, mit der Ziffer 5 beginnen und die letzte Ziffer darf keine 0 sein
  \item Als Größe soll „Large“, „Extra Large“ oder „XX Large“ angegeben werden können.
  \item Durch einen regulären Ausdruck soll sichergestellt werden, dass eine Uhrzeit (24 Stunden, 00 bis 23) im Format hh:mm:ss angegeben wird. Die Angabe der Sekunden soll dabei optional sein.
  \item Das Datum soll in folgendem Format dargestellt werden: dd.mm.yyyy. Der Tag soll dabei zwischen dem 01. und 15 liegen und das Jahr zwischen 1990 und 2010.
\end{itemize}


\subsubsection*{Lösung}
\label{subsubsec:uebung_08.aufgabe_03.loesung}
\inputawk{./awk/uebung_08_-_aufgabe_04.awk}
